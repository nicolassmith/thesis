% $Log: abstract.tex,v $
% Revision 1.1  93/05/14  14:56:25  starflt
% Initial revision
% 
% Revision 1.1  90/05/04  10:41:01  lwvanels
% Initial revision
% 
%
%% The text of your abstract and nothing else (other than comments) goes here.
%% It will be single-spaced and the rest of the text that is supposed to go on
%% the abstract page will be generated by the abstractpage environment.  This
%% file should be \input (not \include 'd) from cover.tex.

The detection of gravitational waves (GWs) from astrophysical sources shows promise as a new method to probe extremely energetic phenomena and test the strong field limit of the general theory of relativity. %NL%
The era of the first generation of broadband interferometric GW antennae is now drawing to a close, and the construction of the second generation has begun. %NL%
The Laser Interferometer Gravitational-wave Observatory (LIGO) in the United States is one component of a worldwide array of sites designed to collectively record and analyze these GW signals. %NL%
In preparation for the next major phase of operation, named Advanced LIGO, an incremental upgrade and prototyping project known as Enhanced LIGO introduced several upgrades to the initial LIGO detectors. %NL%
The addition of the output mode cleaner (OMC), a critically coupled optical cavity designed to filter undesired light from the output of the interferometer before the GW signal is sensed on a photodetector, was one of these upgrades.

This works describes several lessons learned as a result of the installation and commissioning of the OMC in Enhanced LIGO. %NL%
The techniques described in this thesis include the development of a novel OMC alignment system designed to maximally transmit the GW signal in the presence of contamination that would confound a typical automatic alignment system, a design for a remotely controllable automatic mode matching system for the OMC, and prescriptions for reducing the presence of beam jitter noise associated with the OMC. %NL%
The designs of each of the future GW detectors include the use of an OMC, thus the techniques described in this thesis will be directly applicable to achieving the maximum sensitivity of these detectors.
