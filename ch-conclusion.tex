\chapter{Conclusion}
\label{ch:conclusion}

The beginning of the second generation of large scale interferometric gravitational wave antennae has begun. %NL%
A number of new interferometers are currently in various stages of construction. %NL%
Advanced LIGO in the United States \cite{aLIGO}, with possibly a third site in India \cite{M1100296}, Advanced Virgo in Italy \cite{aVirgo}, GEO-HF in Germany \cite{GEOHF}, and KAGRA in Japan \cite{KAGRA}, will comprise a worldwide network of detectors with sensitivity expected to yield several gravitational wave signals of astrophysical origin per year \cite{CBCrates}. %NL%
Achieving the maximum possible sensitivity of these detectors promises to increase the frequency of the detection of signals, and an improvement of the quality of science that can be done with them. %NL%
Each one of these future detectors will employ an output mode cleaner and stands to benefit from the techniques described in this thesis. %NL%
Indeed, this benefit will only increase as squeezed light injection becomes common practice among these detectors.

Many of the difficulties experienced with the output mode cleaner in Enhanced LIGO are already being addressed. %NL%
Changes to the design of the OMC for Advanced LIGO include using a long range PZT for length actuation, eliminating the need for the thermal length actuator (OTAS) and removing the associated issues with small beam clearance in the cavity and thermal dependence of the cavity $g$-parameter \cite{T1000276,T0900157}. %NL%
Also, the tombstones themselves will be polished and coated to act as the input and output couplers, without the need of separate optic to be bonded. %NL%
 

The modal content of the output beam of an interferometer is typically very difficult to model, though there are reasons to believe that the amount of junk light in Advanced LIGO will be reduced compared to Enhanced LIGO. %NL%
Advanced LIGO will employ stable power recycling and signal extraction cavities, so higher order modes will be suppressed \cite{T080208}. %NL%
For any junk light that remains, an optimal alignment scheme such as the one described in Chapter \ref{ch:beacon} will allow maximum signal transmission through the OMC.

Enhanced LIGO further cemented the importance of beam jitter noise when using an OMC. %NL%
Advanced LIGO will employ much advanced seismic isolation platforms \cite{BSCISI} and suspension systems \cite{quaddesign} for the main interferometer test mass optics compared to Enhanced LIGO. %NL%
This will lead to much reduced low frequency beam motion, and consequently its contribution to beam jitter noise. %NL%
In addition, the Tip Tilts have been completely redesigned to employ blade springs from the outset, as well as to have longer suspension wires, increasing vibration isolation. %NL%
In any case, commissioners of Advanced LIGO should expect beam jitter to rear its head sooner or later, and be prepared with the tools to combat it (for example, those described in Chapter \ref{ch:jitter}).

One of the largest sources of output loss for the H1 interferometer in Enhanced LIGO was mode matching \cite{Tobin}. %NL%
The use of a mode matching servo (as described in Chapter \ref{ch:modematching}) to optimize the signal transmission will minimize such losses in future interferometers.

After most of the primal human needs have been met, after a person has shelter, food, and love, there is one need which will be difficult to ever fully satisfy. %NL%
This is the need to understand, to understand oneself, to understand the Universe and one's place therein. %NL%
Just as carpenters build shelter and farmers provide food, scientists must provide for that need of people to understand. %NL%
Every theory and discovery may not feed the popular need to understand, but once in a while, a simple idea can vastly alter the state of human understanding. %NL%
All materials found on Earth are the product of approximately 90 unique ingredients. %NL%
Every living creature is related to every other in a single family tree. %NL%
The Universe as we know it had a beginning, and it occurred around 14 billion years ago. %NL%
These ideas, while shockingly profound, fit inside of a fortune cookie. %NL%
They are the product of countless hours of tedium, missteps, insight, and glory on the part of scientists. %NL%
The author believes that gravitational wave astronomy, while currently in its pre-infancy, will eventually provide such new lessons and discoveries. %NL%
Looking deep into space, one sees a huge wall of fire, which occurred hundreds of thousands of years after the Big Bang, through which no photon may pass unscathed. %NL%
To see deeper, we will need a new messenger. %NL%
Gravitational waves show promise to be our messengers, and this author is honored to be a part of the early efforts to finally grasp these elusive vibrations of space and time.
