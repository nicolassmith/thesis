\usepackage{amsmath} % so inkscape can use this as preamble file
% how to make vectors etc
\newcommand{\vect}[1]{\left| #1 \right> }
\newcommand{\form}[1]{\left< #1 \right| }
\newcommand{\geom}[1]{\vec{#1}}
\newcommand{\oper}[1]{\mathsf{#1}}
%\newcommand{\inprod}[2]{\left< #1 \vphantom{#2} \right|
% \left. #2 \vphantom{#1} \right>}
\newcommand{\inprod}[2]{\left< #1 \vphantom{#2} \right|
 \left. \! \! \! \: #2 \vphantom{#1} \right>}
%\newcommand{\inprod}[2]{\left. \left< #1 \vphantom{#2} \right. \right|
% \left. #2 \vphantom{#1} \right>}
\newcommand{\matrixel}[3]{\left< #1 \vphantom{#2#3} \right|
  \! #2  \! \left| #3 \vphantom{#1#2} \right>}

% notation shorthand
\newcommand{\TEM}[1]{${\rm TEM}_{#1}$}
\newcommand{\dd}{\; \mathrm{d}}
\newcommand{\dt}{\mathrm{d}}
\newcommand{\perc}{\%}
\newcommand{\degrees}{\ensuremath{^\circ}}

% comment
\newcommand{\com}[1]{{\color{red} comment: #1}}

% matricies
\newcommand{\ms}{\left[\begin{matrix}}
\newcommand{\me}{\end{matrix}\right]}

% infobox!
%%%% Custom Command for floating Infoboxes
%%%% usage: \infobox{<title>}{<text>}
\newcommand{\infobox}[2]{
    \parpic(0.4\textwidth,0pt)[rf]{
        \parbox[b]{0.38\textwidth}{
             \bigskip {\bf #1}  \small{{{\sffamily #2}}} \bigskip
        }
    }
    \bigskip
}
