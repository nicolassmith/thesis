\chapter{Gravitational wave antenn\ae{}}

\section{Experimental efforts to detect gravitational waves}
In 1960 Joseph Weber suggested using the resonance of an aluminum bar as an antenna for gravitational waves \cite{bar1}. %NL%
The idea is that a passing short gravitational
wave pulse would induce a strain in the bar and excite its resonance. %NL%
For reasons detailed elsewhere\cite{bar2,bar3}, the community was never able to verify Weber’s subsequent claims of detection\cite{bar4} although various theories\cite{bar5,bar6} were developed to explain the enormous apparent flux of gravitational wave energy.

Since Weber’s pioneering work, resonant bar detectors have come a long way. %NL%
Today’s bar detectors are cryogenically cooled, have much improved seismic isolations and make use of SQUIDs to readout the signal \cite{bar7}.

Using light waves to measure gravitational pertubations was first pointed out by Pirani in 1956 \cite{ifo1} and by 1971 a prototype interferometer was built in Malibu \cite{ifo2}. %NL%
Shortly after that a study done at MIT by R. %NL%
Weiss identified almost all noise sources relevant for a kilometer length scale interferometer \cite{ifo3}. %NL%
It is these broadband laser interferometer based gravitational wave antenn\ae{} which provide the most compelling chance to directly measure these elusive waves.

\section{Measurement of optical phase}
In Chapter \ref{ch:gws} we introduced the idea that the waveform of a passing gravitational wave is imprinted on the phase of photons. %NL%
The frequency of ocillation of a laser with a wavelength of 1.064$\upmu$m is 2.8$\times 10^{14}$Hz. %NL%
There is no technology, currently concieved, which can directly measure the phase of electromagnetic waves at optical frequencies. %NL%
One may, however, convert the change of phase of the electromagnetic wave into a change of amplitude by means of the phenomenon of interference. %NL%
This is acheived by superposing two electromagnetic waves and measuring the power of the resulting wave, which is straightforward withe use of a photodetector (PD). %NL%


A device which exploits interference to measure phase changes of light is refered to as an \emph{interferometer}. %NL%
Modern interferometers use one or multiple lasers as the source of light. %NL%
Laser light is very well equipped for interferometry. %NL%
Laser light is highly phase coherent compared to other light sources. %NL%
Also, a laser beam can often approach the minimum beam divergence physically allowed. %NL%
Types of interferometers include: Fabry-Perot, Michelson, Mach-Zehnder etc.

\section{Resonant optical cavity}
A resonant cavity, sometimes refered to as a Fabry-Perot (FP) cavity, is constructed with two partially transmitting mirrors placed in series. %NL%
When light is incident on the front, or input, mirror some is transmitted into the cavity. %NL%
As a steady state is approached, the light already in the cavity may interfere with new light which is being pumped through the input mirror. %NL%
When the circulating field is in phase with the incident pumping field, \emph{resonance} occurs. %NL%
For a linear cavity, this happens when the length is equal to an integer number of half-wavelengths of the light \cite{Siegman}.

\com{introduce the idea of over/under/critically coupled}

\section{Pound-Drever-Hall reflection locking}
\com{PDH}\cite{PDH}
Pound-Drever-Hall (PDH) reflection locking is a powerful technique by which the resonance condition of a laser incident on an optical cavity may be controlled by use of a feedback control system. %NL%
In this control system, the error signal is a measurement of the detuning of the laser light from resonance. %NL%
In the case of a fixed cavity, this may be understood as a measurement of the fluctuations in the laser wavelength. %NL%
We are instead concerned with a very stable laser source and a cavity which may change length due to external influences. %NL%
In this case, the error signal is better understood as a measurement of the length changes of the cavity.

\com{phase modulation}

\com{frontal phase modulation}

\section{The Michelson interferometer}
\com{common mode rejection}

\com{common mode servo}

\com{90\degree{} is good for GWs}

\section{Power recycling}
\com{Power enhancement}

\com{coupled-cavity carrier filtering}

\com{somewhere a figure of all LIGO}

\section{The benefits of non-modulated readout}

\com{end with segue to OMC ch. For example, show how losses at the output reduce SNR.}
