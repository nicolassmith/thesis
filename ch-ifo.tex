\chapter{Gravitational wave antenn\ae{}}

\section{Experimental efforts to detect gravitational waves}
In 1960, Joseph Weber made the suggestion that an aluminum bar could be used as a gravitational wave antenna \cite{bar1}. %NL%
The idea being that as a burst of gravitational wave passes, it would induce a strain in the bar and excite the resonant mechanical modes of the bar. %NL%
For reasons detailed elsewhere\cite{bar2,bar3}, the scientific community was never able to verify Weber’s subsequent claims of detection\cite{bar4} although various theories\cite{bar5,bar6} were developed to explain the enormous apparent flux of gravitational wave energy.

In the subsequent years following Weber’s pioneering work, resonant bar detectors have improved greatly. %NL%
Modern bar detectors are cryogenically cooled, have much improved seismic isolation and make use of SQUIDs to readout the signal \cite{bar7}.

Using light waves to measure gravitational pertubations was first pointed out by Pirani in 1956 \cite{ifo1}. %NL%
It was only until 1971 that an early prototype interferometer was built in Malibu using an audio cassete recorder for data acquisition \cite{ifo2}. %NL%
Shortly after that, a study done at MIT by R. %NL%
Weiss identified almost all noise sources relevant for a kilometer length scale interferometer \cite{ifo3}. %NL%
It is these broadband laser interferometer based gravitational wave antenn\ae{} which provide the most compelling chance to directly measure these elusive waves.

\section{Measurement of optical phase}
In Chapter \ref{ch:gws} we introduced the idea that the waveform of a passing gravitational wave is imprinted on the phase of photons. %NL%
The frequency of ocillation of a laser with a wavelength of 1.064$\upmu$m is 2.8$\times 10^{14}$Hz. %NL%
There is no technology, currently concieved, which can directly measure the phase of electromagnetic waves at optical frequencies. %NL%
One may, however, convert the change of phase of the electromagnetic wave into a change of amplitude by means of the phenomenon of interference. %NL%
This is acheived by superposing two electromagnetic waves and measuring the power of the resulting wave, which is straightforward withe use of a photodetector (PD). %NL%


A device which exploits interference to measure phase changes of light is refered to as an \emph{interferometer}. %NL%
Modern interferometers use one or multiple lasers as the source of light. %NL%
Laser light is very well equipped for interferometry. %NL%
Laser light is highly phase coherent compared to other light sources. %NL%
Also, a laser beam can often approach the minimum beam divergence physically allowed. %NL%
Types of interferometers include: Fabry-Perot, Michelson, Mach-Zehnder etc.

\section{Resonant optical cavity}
A resonant cavity, sometimes refered to as a Fabry-Perot (FP) cavity, is constructed with two partially transmitting mirrors placed in series. %NL%
When light is incident on the front, or input, mirror some is transmitted into the cavity. %NL%
As a steady state is approached, the light already in the cavity may interfere with new light which is being pumped through the input mirror. %NL%
When the circulating field is in phase with the incident pumping field, \emph{resonance} occurs. %NL%
For a linear cavity, this happens when the length is equal to an integer number of half-wavelengths of the light \cite{Siegman}.

\com{introduce the idea of over/under/critically coupled}

\section{Pound-Drever-Hall reflection locking}
Pound-Drever-Hall (PDH) reflection locking is a powerful technique by which the resonance condition of a laser incident on an optical cavity may be controlled by use of a feedback control system \cite{PDH}. %NL%
In this control system, the error signal is a measurement of the detuning of the laser light from resonance. %NL%
In the case of a fixed length cavity, this may be understood as a measurement of the fluctuations in the laser wavelength. %NL%
We are instead concerned with a very stable laser source and a cavity which may change length due to external influences.\footnote{A gravitational wave changes the phase accumulated in the cavity, which is essentially interchangable with the optical path length of the cavity.} In this case, the error signal is better understood as a measurement of the length changes of the cavity.

As was shown in the previous section, the phase shift of the beam reflected from the resonant cavity is enhanced for fields resonant in the cavity. %NL%
However, for fields not resonant in the cavity, there is nearly no phase shift of the reflected beam. %NL%
The PDH sensing technique exploits this fact by contructing an input beam with frequency components that are both resonant and non-resonant. %NL%
By doing a relative phase measurement of the reflected fields, one may determine length changes of the cavity.

Creation of the input beam is done by sending the beam through an optical modulator. %NL%
This usually comes in the form of an \emph{electro optic modulator} (EOM), which is an optical device that has a electronically variable optical path length. %NL%
Applying a periodic electronic signal, usually at radio frequencies (RF), will induce a periodic phase modulation of the laser beam. %NL%
As will be discussed in Section \ref{sec:freqspace}, the primary result of this phase modulation is to impose new optical fields separated in frequency from the original field by the RF modulation frequency. %NL%
These new fields are usually refered to as \emph{phase modulated sidebands}, the central frequency component is refered to as the \emph{carrier}.\com{figure with carrier/sidebands and cavity?} The modulation frequency is chosen such that the sidebands are not resonant in the cavity, and thus do not experience a phase shift when the cavity changes length. %NL%
Comparison of the constant sideband phase with the resonating carrier phase through interferometry yields a very sensitive cavity length measurement. %NL%


\com{it's nice that they start as phase sidebands, so there is no signal when perfectly resonating (resonance is actully 180deg phase shift}

\com{moving from resonance rotates the carrier phase, and thus turns phase sidebands into amplitude sidebands, which produce a power modulation}
\com{heterodyne detection scheme is relative phase measurement}

\section{The Michelson interferometer}
\com{common mode rejection, with very simple calculation that has contrast defect}

\com{common mode servo}

\com{90\degree{} is good for GWs}

\section{Power recycling}
\com{Power enhancement}

\com{coupled-cavity carrier filtering}

\com{somewhere a figure of all LIGO}

\section{The benefits of non-modulated readout}

\com{end with segue to OMC ch. For example, show how losses at the output reduce SNR.}
