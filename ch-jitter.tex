\chapter{Techniques for reducing output mode cleaner beam jitter noise}
%This was in my opinion the most important noise source that came with the inclusion of the OMC in LIGO

\section{The noise mechanism of beam jitter incident on a high finesse cavity}
Motion of the beam incident on a high finesse cavity alters the overlap of said beam with the resonant mode of the cavity. %NL%
This leads to fluctuations in the amount of light which can couple into the cavity and consequently, power fluctuations of the transmitted beam. %NL%
The power fluctuation can spoil the measurement of the intrinsic amplitude modulation of the beam if the measurement and fluctuations occur at the same frequencies.

For a beam incident on the cavity which has a lateral (perpendicular to the direction of propagation) and angular misalignment, constrained to a plane, we have the following expression for the power transmitted through the cavity
\begin{equation}
\label{eqn:simplejitter}
P\approx P_0\left(1-\left(\frac{\Delta x}{w_0}\right)^2-\left(\frac{\theta}{\theta_d}\right)^2\right),
\end{equation}
where $P_0$ is the power transmitted of the aligned beam, $\Delta x$ is the beam waist displacement, $\theta$ is the beam waist tilt, $w_0$ is the beam waist radius, and $\theta_d$ is the divergence angle of the beam. %NL%
The approximation holds while both the angular and lateral misalignments are small.

Equation \ref{eqn:simplejitter} shows that beam misalignments couple to the transmitted power quadradically. %NL%
This leads to a few relevant consequences. %NL%
The coupling of beam jitter to transmitted power is nonlinear, thus frequency components of the beam motion may be mixed together, producing new frequencies in the transmitted power. %NL%
Also, the linear coupling coefficient of beam jitter to power fluctuations is proportional to the DC beam misalignment. %NL%
This causes the direct linear coupling of frequency components of beam jitter to power fluctuations to vary as the DC pointing error varies.

Steering optics in the beam path leading to the cavity which are vibrating may transfer vibrations to the laser beam and produce beam jitter noise. %NL%
Let us investigate beam jitter coupling of an optic in the modal picture. %NL%
 We will work in a basis of only the \TEM{00} and \TEM{01} modes. %NL%
Let's assume we have a beam which is dominantly in the \TEM{00} mode as measured in the basis of our cavity, with some small amplitude of \TEM{01} mode. %NL%
In other words, we begin with a beam which is slightly misaligned.
\begin{equation}
\vect{E}_{\text{input}}=E_0\left(\vect{00}+\alpha\vect{01}\right).
\end{equation}
The misalignment, $\alpha$ may be do to another steering mirror upstream, or a misalignment of the beam source relative to our cavity axis, for this discussion it is not relevant.

Now we will assume this beam is incident on a steering mirror before it finally enters the cavity. %NL%
This steering mirror also has a slight misalignment, $\theta$. %NL%
Equation \ref{eqn:mirrortilt} shows the matrix representing the operator of a mirror which has been tilted from nominal alignment. %NL%
The beam incident on the cavity is then
\begin{equation}
\vect{E}_{\text{incident}}=\oper{M}(\Theta)\vect{E}_{\text{input}},
\end{equation}
where, as before, $\Theta=\frac{\theta \pi w(z)}{\lambda}$, while $w(z)$ is the beam width radius at the mirror, and $\lambda$ is the laser wavelength. %NL%
Because the cavity has a high finesse, when the cavity is resonant on the \TEM{00} mode, only that mode is transmitted. %NL%
Thus, for the field transmitted by the cavity
\begin{equation}
E_{\text{t}}=t_{\text{cavity}}\matrixel{00}{\oper{M}\left(\Theta\right)}{E}_{\text{input}}=t_{\text{cavity}}E_0\left(1-i\frac{\alpha \pi w(z)}{\lambda}\theta\right),
\end{equation}
where \com{finish}. %NL%
This causes a power fluctuation on trasnmission
\begin{equation}
\label{eqn:mirrorjitter}
P=E_t^*E_t=P_0\left(1+\frac{2\pi w(z)}{\lambda}\theta\Im(\alpha)\right).
\end{equation}
Only the imaginary part of $\alpha$ contributes to beam jitter noise of our mirror because it represents an error in alignment \emph{angle}, not position, at the location of the steering mirror $\oper{M}$. %NL%


One interesting consequence of Equation \ref{eqn:mirrorjitter} is that for a given mirror, the strength of beam jitter coupling is proportional to $w(z)$, the size of the beam on the optic. %NL%
Thus one technique for reducing beam jitter coupling is to design one's optical path in such a way that beams are smaller on optics that may produce significant beam jitter. %NL%
This of course also reduces their efficiency as an alignment control mirror.

Futhermore, we again see a nonlinear coupling in the form of mixing of the incident misalignment, $\alpha$, and the jitter of the steering mirror, $\theta$. %NL%
We may suppose that the input alignment has both a DC component and possibly some low frequency (a few to several Hz) wandering motion. %NL%
If this is coupled with high frequency (i.e. %NL%
frequencies in the detection band) jitter of the steering mirror, the mixing of frequencies can cause noise in the power transmission at and around the jitter frequency of the steering optic.

\com{introduce bilinear beam jitter in s6}
\com{show plot similar to tobin's about bilinear coupling}

\section{Mechanical resonances of beam steering optics}
% Where does HAM6 come into all this?
The Enhanced LIGO upgrade introduced several modifications to the readout chain of the interferometer. %NL%
As discussed in Chapter \ref{ch:omc}, with the introduction of the in-vacuum OMC, it was necessary to use new beam steering and modematching optics to couple the beam exiting the interferometer to the OMC cavity. %NL%
These steering optics, as well as the OMC, were all housed in a single vacuum chamber on top of the prototype HAM ISI platform which is planned to be used copiously in Advanced LIGO. %NL%
Although it was not fully appreciated at the start of Enhanced LIGO, the new isolation platform provided much reduced vibration isolation at audio frequencies when compared to the isolation tables used in the HAM chambers in Initial LIGO. %NL%
\com{Jeff's figure}

% Talk about comparison of HAM ISI vs HAM stack

% show original TT0 (mirror on post)

% original TT design, no blades, fat wires

% skinny wires

% blades

% figure showing sensitivity improvement with each susp.

\section{Magnetic field coupling and feedforward subtraction}
% if you have aux sensors of beam jitter mechanism, you can perform subtraction.

% show figure of performance

% reference to sergey's awesome subtraction stuff

\section{The relationship of alignment control and beam jitter}
% depending on bandwidth, alignment system can remove low/DC or high. Though not if you are using a beacon system. Also beware of fringe wrapping.
As discussed in Chapter \ref{ch:beacon}, a standard cavity alignment control system is designed to minimize the \TEM{01} and \TEM{10} mode content incident on the cavity. %NL%
We have also just seen that any misalignment increases the linear coupling of beam jitter noise to the cavity transmission signal. %NL%
This connection is manifestly clear in the case of a dither alignment servo, where beam jitter is purposely induced on the beam, and the servo obtains optimum alignment by minimizing the coupling of the dither frequency in transmission. %NL%
If the DC and low frequencies of the alignment error can be sufficiently supressed by a control system, the beam jitter coupling is consequently reduced.

%Attempts were made in Enhanced LIGO to directly reduce the high frequency beam jitter on the OMC with a high bandwidth alignment servo. For sufficiently high control bandwith, it was seen that an increase in gain would cause an increase in broadband noise in the OMC transmission. One possible mechanism for the increased noise \com{OMC backscatter} \cite{T060303}.

We break this relationship between alignment optimization and jitter coupling minimization when a beacon or optimal style alignment system is used as desribed in Section \ref{sec:beaconalignment}. %NL%
Because the carrier frequency component is often the dominant contribution to the total power of the transmitted signal, it is usually the case that beam jitter noise orignating from the carrier is dominant over other frequency components. %NL%
Alignment schemes which optimize the alignment of the audio frequency signal sidebands will disregard the HOM content of the carrier beam, and thus may increase beam jitter coupling. %NL%
This additional beam jitter coupling may be mitigated by control of the relative modal content of the audio sidebands and the carrier, which will be discussed in the following section.

\section[Modification of modal content of the carrier and audio sidebands]{Modification of modal content of the carrier and audio sidebands\footnote{This section will satisfy as mearly a record of some lore learned during Enhanced LIGO. It is not intended to provide a quantitative model, but merely act as a guide for future efforts.}}

After the implementation of the beacon alignment system in Enhanced LIGO, some beam jitter peaks of unknown origin remained in the sensitive region of the detector. %NL%
\com{figure?}. %NL%
It was discovered that the height of these peaks were controlable by modifying the beam position on the antisymmetric port alignment sensor\footnote{In Initial/Enhanced LIGO, this was known as WFS1.} of the main interferometer alignment control system. %NL%
We postulated that because of offsets in the readout of the interferometer alignment signals, changing the beam position would change the error point offsets of the alignment feedback system of the main interferometer. %NL%
The antisymmetric port signal primarily feeds back to the differential alignment mode of the ETMs. %NL%
It was thought that this lead to a redistribution of the modal content of the carrier field exiting the antisymetric port of the interferometer relative to the fields of the audio sidebands. %NL%
One could imagine that the position of a local minimum, with respect to alignment, of the carrier transmission could be moved into coincidence with the optimal alignment point of the signal fields. %NL%
Attempts were made to instead directly inject offsets into the alignment error signals to produce the same effect, but they were never as successful as varying the pointing on the alingment sensor.
