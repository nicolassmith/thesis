
%%% Local Variables: 
%%% mode: latex
%%% TeX-master: "main"
%%% End: 

\chapter{Modal Model of a Laser Field}

This chapter will describe a formalism which treats the laser field components propagating in an optical interferometer as a vector space where common optical components, such as mirrors and optical modulators are treated as operators. %NL%
This treatment allows the analysis of complicated optical paths, which include paths that loop onto themselves (as in the case of resonant cavities) to be reduced to a problem of matrix manipulation. %NL%
This is is highly powerful technique which is used extensively in modeling of gravitational-wave interferometers.

To just get a flavor of the formalism, consider the simple example where we choose our vector space to just have three components which represent the amplitude of a carrier frequency field at frequency $\omega_0$, as well as two sideband fields separated at $\pm\Omega$. %NL%
Thus, if the electric field of a propagating electromagnetic wave along the propagation axis ($z$) is written as
\begin{align*}
\geom{E}(z,t) &= A_0 e^{i[\omega_0t -kz]}\geom{x} + A_1e^{[i(\omega_0+\Omega)t -kz]}\geom{x} + A_{-1}e^{i[(\omega_0-\Omega)t -kz]}\geom{x}\\
&= \left( A_0 + A_1 e^{i\Omega t}+A_{-1}e^{-i\Omega t}\right) e^{i[\omega_0 t-kz]}\geom{x},
\end{align*}
where $A_0$, $A_1$ and $A_{-1}$ are the carrier, first order upper sideband and first order lower sideband amplitudes respectively, $k=\omega_0/c$ is the wavenumber, and $\geom{x}$ is the \emph{geometrical} basis vector poiting along the axis of polarization, then we are interested in the vector representation where this same propagating field is represented as\footnote{The use of $\doteq$ in (\ref{eq:dotequals}) is meant to convey that the expression on the right side is a representation of the vector on the left side in the chosen vector space. %NL%
The right side of the equation is not written in terms of vectors and thus is not truly equal to the left side. %NL%
The dot can be ignored if one doesn't mind their math to be a bit sloppy.} 
\begin{equation}
\label{eq:dotequals}
\vect{E} \doteq \ms A_0 \\ A_1 \\ A_{-1} \me.
\end{equation}

By inspection it is clear that our representation uses the following basis vectors to span our vector space:
\begin{align}
\vect{0}&\doteq e^{i[\omega_0 t-kz]}\geom{x}\notag, \\
\vect{1}&\doteq e^{i\Omega t}e^{i[\omega_0 t-kz]}\geom{x}, \\
\vect{-1}&\doteq e^{-i\Omega t}e^{i[\omega_0 t-kz]}\geom{x}\notag.
\end{align}
It is important to emphesize that the vectors used in this formalism are in general not simple euclidian vectors, but rather abstract vectors representing different components of the electric field of an electromagnetic wave. %NL%
In this simple case, the main aspect which distinguishes the three basis vectors is the frequency of periodic oscillations. %NL%


Now that we are familiar with using this vector space to represent the desired wave amplitudes, let's examine how we may use an operator to represent some optical component. %NL%
An electro-optic modulator (EOM) can act as a phase modulator for a laser field when a voltage is applied. %NL%
When a periodic signal is applied at frequency $\Omega$, given an input field amplitude $Ae^{i\omega_0 t}$ (ignoring the $z$ dependence) the output amplitude is
\newcommand{\gammahalf}{\frac{i\Gamma}{2}}
\begin{equation}
\label{eq:inputfield}
Ae^{i\omega_0 t + i\Gamma \cos{\Omega t}}\approx Ae^{i\omega_0 t}\left(1+\gammahalf e^{i\Omega t}+\gammahalf e^{-i\Omega t}\right),
\end{equation}
where $\Gamma$ is known as the modulation depth and is assumed to be small. %NL%
One can show that in the example vector space we are considering, this EOM can be represented as the following operator
\begin{equation}
\oper{M_\phi} \doteq 
\ms 
1          & \gammahalf & \gammahalf \\
\gammahalf & 1          & 0          \\
\gammahalf & 0          & 1 \\
\me.
\end{equation}
It is then clear that the act of $\oper{M_\phi}$ operating on an input field constructed as the one used in (\ref{eq:inputfield}) gives the desired output field:
\begin{equation}
\oper{M_\phi}\vect{E_{\text{input}}} \doteq \ms 
1          & \gammahalf & \gammahalf \\
\gammahalf & 1          & 0          \\
\gammahalf & 0          & 1 \\
\me
\ms A \\ 0 \\ 0 \me = A \ms  1 \\ \gammahalf \\ \gammahalf \me.
\end{equation}

So far, we have kept our example relatively simple. %NL%
The vector space we have chosen represents only the first order sidebands generated on the carrier frequency. %NL%
Also, we have only expanded the effect of phase modulation to terms linear in $\Gamma$. %NL%
There is, however, no fundamental reason that the analysis cannot be used to arbitrary precision. %NL%
The number of components of the vector space can be expanded to account for an arbitrary number of higher order sidebands, and the matrix elements used for $\oper{M_\phi}$ can take their exact values. %NL%
The general form of the phase modulation operator is\footnote{This can be shown using the so-called Jacobi-Anger identity, $e^{ix\cos\theta}=\sum_{k=-\infty}^\infty i^kJ_k(x)e^{ik\theta}$.}:
\begin{equation}
\oper{M_\phi} = \sum_{r,s=-\infty}^\infty\vect{r} i^{s-r} J_{s-r}(\Gamma) \form{s},
\end{equation}
where $\vect{r}$ is the basis vector representing the $r$th order sideband component (which includes a $e^{ir\Omega t}$ time dependence), and $J_k(x)$ is the bessel function of the first kind. %NL%
Also, $J_{-k}(x)=(-1)^kJ_k(x)$. %NL%
The ability to calculate to arbitrary precision will continue to hold true when we expand the treatment to include other components of a propagating laser field, such as the transverse beam modes. %NL%
As long as the solutions are converging, the precision is only limited by the choice of when the vector space is truncated, i.e. %NL%
to what order the calculation the calculation is taken.