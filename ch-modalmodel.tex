
%%% Local Variables: 
%%% mode: latex
%%% TeX-master: "main"
%%% End: 

\chapter{Vector Space Model of Optical Systems}

\section{Introduction}
This chapter will describe a formalism which treats the laser field components propagating in an optical interferometer as a complex vector space where common optical components, such as mirrors and optical modulators are treated as operators in this space. %NL%
This treatment allows the analysis of complicated optical paths, which include paths that loop onto themselves (as in the case of resonant cavities) to be reduced to a problem of matrix manipulation. %NL%
This is is highly powerful technique which is used extensively in modeling the behavior of gravitational-wave interferometers \cite{Vinet1986,Hefetz:97,Sigg:00}.

\section{Frequency components of a laser field}
\label{sec:freqspace}
To just get a flavor of the formalism, consider the simple example where we choose our vector space to just have three components which represent the amplitude of a carrier frequency field at frequency $\omega_0$, as well as two sideband fields separated in frequency by $\pm\Omega$. %NL%
Thus, if the electric field of a propagating electromagnetic wave along the propagation axis ($z$) is written as
\begin{align*}
\geom{E}(z,t) &= A_0 e^{i[\omega_0t -kz]}\geom{x} + A_1e^{[i(\omega_0+\Omega)t -kz]}\geom{x} + A_{-1}e^{i[(\omega_0-\Omega)t -kz]}\geom{x}\\
&= \left( A_0 + A_1 e^{i\Omega t}+A_{-1}e^{-i\Omega t}\right) e^{i[\omega_0 t-kz]}\geom{x},
\end{align*}
where $A_0$, $A_1$ and $A_{-1}$ are the carrier, first order upper sideband and first order lower sideband amplitudes respectively, $k=\omega_0/c$ is the wavenumber, and $\geom{x}$ is the \emph{geometrical} basis vector pointing along the axis of polarization, then we are interested in the vector representation where this same propagating field is represented as\footnote{The use of $\doteq$ in (\ref{eq:dotequals}) is meant to convey that the expression on the right side is a representation of the vector on the left side in the chosen vector space. %NL%
The right side of the equation is not written in terms of vectors and thus is not truly equal to the left side. %NL%
The dot can be ignored if one doesn't mind their math to be a bit sloppy.} 
\begin{equation}
\label{eq:dotequals}
\vect{E} \doteq \ms A_0 \\ A_1 \\ A_{-1} \me.
\end{equation}

By inspection it is clear that our representation uses the following basis vectors to span our vector space:
\begin{align}
\vect{0}&\doteq e^{i[\omega_0 t-kz]}\geom{x}\notag, \\
\vect{1}&\doteq e^{i\Omega t}e^{i[\omega_0 t-kz]}\geom{x}, \\
\vect{-1}&\doteq e^{-i\Omega t}e^{i[\omega_0 t-kz]}\geom{x}\notag.
\end{align}
It is important to emphasize that the vectors used in this formalism are in general not simple euclidean vectors, but rather abstract vectors representing different components of the electric field of an electromagnetic wave. %NL%
In this simple case, the main aspect which distinguishes the three basis vectors is the frequency of periodic oscillations. %NL%
The inner product of two basis vectors is:
\[
\inprod{r}{s} = \int\limits_{\text{many cycles}}\! \! \! \! \! \! \! \! \!
\dd t \; \: {\left( e^{i\omega_{r}t}\right)^*}e^{i\omega_{s}t} = \delta_{rs},
\]
where $\omega_r$ is the sideband frequency of the $r$th sideband. %NL%
This treatment is analogous to the formalism of quantum mechanics in which the quantum state of a system is defined by a vector in a space spanned by a set of orthonormal basis state vectors. %NL%
In our formalism, the components of the laser field will take the place of the state vectors. %NL%
We will use Dirac bra-ket notation because of this strong analogy.\footnote{One significant difference between quantum mechanics and this formalism is that the quantum state vector of a system must always have unit length, to ensure the state has a probability of 1. %NL%
The length of the vector in this formalism is just the electric field amplitude, and can take any value.}

Now that we are familiar with using this vector space to represent the desired wave amplitudes, let's examine how we may use an operator to represent some optical component. %NL%
An electro-optic modulator (EOM) can act as a phase modulator for a laser field when a voltage is applied. %NL%
When a periodic signal is applied at frequency $\Omega$, given an input field amplitude $Ae^{i\omega_0 t}$ (ignoring the $z$ dependence) the output amplitude is
\newcommand{\gammahalf}{\frac{i\Gamma}{2}}
\begin{equation}
\label{eq:inputfield}
Ae^{i\omega_0 t + i\Gamma \cos{\Omega t}}\approx Ae^{i\omega_0 t}\left(1+\gammahalf e^{i\Omega t}+\gammahalf e^{-i\Omega t}\right),
\end{equation}
where $\Gamma$ is known as the modulation depth and is assumed to be small. %NL%
One can show that in the example vector space we are considering, this EOM can be represented as the following operator
\begin{equation}
\oper{\Phi}(\Gamma) \doteq 
\ms 
1          & \gammahalf & \gammahalf \\
\gammahalf & 1          & 0          \\
\gammahalf & 0          & 1 \\
\me.
\end{equation}
It is then clear that the act of $\oper{\Phi}(\Gamma)$ operating on an input field constructed as the one used in (\ref{eq:inputfield}) gives the desired output field:
\begin{equation}
\vect{E_{output}}=
\oper{\Phi}(\Gamma)\vect{E_{\text{input}}} \doteq \ms 
1          & \gammahalf & \gammahalf \\
\gammahalf & 1          & 0          \\
\gammahalf & 0          & 1 \\
\me
\ms A \\ 0 \\ 0 \me = A \ms  1 \\ \gammahalf \\ \gammahalf \me.
\end{equation}

So far, we have kept our example relatively simple. %NL%
The vector space we have chosen represents only the first order sidebands generated on the carrier frequency. %NL%
Also, we have only expanded the effect of phase modulation to terms linear in $\Gamma$. %NL%
There is, however, no fundamental reason that the analysis cannot be used to arbitrary precision. %NL%
The number of components of the vector space can be expanded to account for an arbitrary number of higher order sidebands, and the matrix elements used for $\oper{\Phi}(\Gamma)$ can take their exact values. %NL%
The general form of the phase modulation operator is:
\begin{equation}
\oper{\Phi}(\Gamma) = \sum_{r,s=-\infty}^\infty\vect{r} i^{s-r} J_{s-r}(\Gamma) \form{s},
\end{equation}
where $\vect{r}$ is the basis vector representing the $r$th order sideband component (which includes a $e^{ir\Omega t}$ time dependence), and $J_k(x)$ is the Bessel function of the first kind.\footnote{This can be shown using the so-called Jacobi-Anger identity, $e^{ix\cos\theta}=\sum_{k=-\infty}^\infty i^kJ_k(x)e^{ik\theta}$. %NL%
Also, $J_{-k}(x)=(-1)^kJ_k(x)$, thus $\oper{\Phi}(\Gamma)$ is symmetric.} The ability to calculate to arbitrary precision will continue to hold true when we expand the treatment to include other components of a propagating laser field, such as the transverse beam modes. %NL%
As long as the solutions are converging, the precision is only limited by the choice of when the vector space is truncated, i.e.\ to what order the calculation is taken.

\section{The modal space}
\label{sec:modalspace}
So far our example prescribes an elegant way to deal with optical components (e.g.\ the modulator) which transfer energy from one component of the electric field (the carrier) to other components (the sidebands). %NL%
This type of approach is also applicable to the components of the electric field corresponding the the transverse spatial modes of the beam, as shown by \citet{Hefetz:97}.

When the paraxial approximation applies it is possible to expand the electromagnetic field into a superposition of modes represented by a Gaussian function of the transverse coordinate, multiplied by polynomials. %NL%
Common choices for the polynomial functions are the Hermite or Laguerre polynomials \cite[chap. %NL%
16]{Siegman}. %NL%
This formalism associates the modal components with eigenmodes of what is described as the ``perfectly aligned and undistorted optical path''. %NL%
In other words, this formalism describes how the introduction of misalignments and beam distortions alter a laser beam propagating in the system by treating misalignments (or higher order beam distortions) as matrix elements which transfer energy between different spatial modes. %NL%
The misalignments are generally treated as small, and thus, the operators of the optical components differ from the identity operator by small corrections. %NL%
The misaligned solutions of beam propagation are thus treated as perturbations of the aligned solutions. %NL%


The treatment of modes in this formalism is slightly novel due to the fact that, normally, some fundamental Gaussian laser mode is defined to have some beam particular waist $w_0$, and a focused beam with a different beam waist can be written as a series of higher order components with the original beam waist $w_0$. %NL%
While in this formalism, the fundamental Gaussian mode may have a beam waist value that changes due to the presence of a lens, but the beam on both sides of the lens would be represented by the same modal state vector (modulo some phase rotation due to propagation). %NL%
It is only the presence of misalignments and distortions that cause significant off-diagonal matrix elements.

Choosing the common Hermite-Gaussian basis, we can use basis vectors which are separable into the mode order of the beam along the $x$ and $y$ transverse axes of the beam. %NL%
A general field vector can be decomposed in the modal space as
\begin{equation}
\vect{E} = \sum_{m,n=0}^\infty A_{mn}\vect{nm}\doteq\sum_{m,n=0}^\infty A_{mn}U_m(x,z)U_n(y,z)e^{-ikz},
\end{equation}
where $U_m(x,z)$ is the $m$th 1-D Hermite-Gauss mode (see \ref{ap:HGmode}). %NL%
The basis modes are also referred to as the Transverse Electric and Magnetic modes of order $mn$ (\TEM{mn}).

\com{Let's also talk about the inner product.}

In the modal space, a simple example operator would be a mirror that has been tilted by an angle $\theta_x$ about the y axis. %NL%
Due to the mirror misalignment, the phase of the beam at the plane of the mirror has been advanced on one side and retarded on the other side, relative to the aligned beam. %NL%
This can be represented as a phase muliplication of $\exp[-2ik\theta_x x]$. %NL%
\citet{Hefetz:97} show that in the modal basis this is satisfied by the operator
\begin{equation}
\oper{M}(\Theta_x) = \exp\left[ -2i\Theta_x\sum_{m,n,k,l=0}^\infty \vect{mn} \delta_{nl} \left( \sqrt k \delta_{m[k-1]}+\sqrt m \delta_{m[k+1]} \right) \form{kl}\right],
\end{equation}
where $\delta_{mn}$ is the Kroneker delta; and $\Theta_x = \theta_x \pi w(z)/\lambda$ is the normalized misalignment angle for beam width $w(z)$ and wavelength $\lambda$, or the ratio of the misalginment angle to the beam divergence angle. %NL%
One can see that the $k$ index is connected to the $m$ index that differs by one mode number, this will provide the correct transfer of energy from the \TEM{00} mode to the \TEM{10} mode as discussed above. %NL%
We can use a matrix representation with three components representing the \TEM{00}, \TEM{10} and \TEM{01} amplitudes, explicitly for field $\vect{E}$ and operator $\oper{O}$:
\begin{align}
\vect{E} &\doteq \ms \inprod{00}{E} \\ \inprod{10}{E} \\ \inprod{01}{E} \me
& \oper{O} &\doteq \ms 
\matrixel{00}{\oper{O}}{00} & \matrixel{00}{\oper{O}}{10} & \matrixel{00}{\oper{O}}{01} \\
\matrixel{10}{\oper{O}}{00} & \matrixel{10}{\oper{O}}{10} & \matrixel{10}{\oper{O}}{01} \\
\matrixel{01}{\oper{O}}{00} & \matrixel{01}{\oper{O}}{10} & \matrixel{01}{\oper{O}}{01} \me
\end{align}

In this representation, 
\begin{equation}
\oper{M}(\Theta_x) \doteq \exp \left( -2i 
\ms 
0 & \Theta_x & 0 \\
\Theta_x & 0 & 0 \\
0 & 0 & 0 
\me
\right)
=
\ms
\cos {2\Theta_x} & -i\sin{2\Theta_x}& 0 \\
-i\sin {2\Theta_x} & \cos{2\Theta_x}& 0 \\
0 & 0 & 1 \\
\me ,
\end{equation}
and for small $\Theta_x$,\footnote{Note that $\theta_x$ being small is also required by the paraxial approximation.}
\[
\approx
\ms
1-2\Theta_x^2 & -2i\Theta_x& 0 \\
-2i\Theta_x & 1-2\Theta_x^2& 0 \\
0 & 0 & 1 \\
\me .
\]
So for small misalignments, the effect of rotating a mirror is to take energy from the \TEM{00} component of the field, and transfer it to the \TEM{10} component, the opposite also being true when the initial field already has a non-zero \TEM{10} component. %NL%
It is also useful to note that the resulting \TEM{10} field has a quadruature which is rotated by $\pi/2$ with respect to the initial \TEM{00} field.

\section{The combined vector space}
To treat the general problem of laser fields propagating through an optical system, it will be useful to consider both frequency and modal components simultaneously. %NL%
This combination was covered in detail by \citet{Sigg:00}.

Mathematically, the combined space is a tensor product of the spaces discussed above. %NL%
The resulting basis vectors will be labeled by four indices, two from the modal space, one from the frequency space, and for completeness, a final index to represent the polarization. %NL%
We can use the following notation for the basis vectors:
\begin{equation}
\vect{mn;r;p}=\vect{mn}_{\text{modal}}\otimes\vect{r}_{\text{frequency}}\otimes\vect{p}_{\text{polarization}}.
\end{equation}
It is understood that the electric field represented by one of these basis vectors is
\[
\vect{mn;r;p} \doteq e^{\left[ i\left( \omega_0+\omega_r\right)t\right]} U_{mn}\geom{\epsilon}_p,
\]
where $\omega_r$ is some mapping from the index $r$ to a set of frequencies; $U_{mn}$ is shorhand for $U_m(x,z)\times U_n(y,z)$; and $\geom{\epsilon}_p$ for $p\in \{1,2\}$ are the polarization unit vectors in the $x$ and $y$ direction, respectively. %NL%
\com{talk about how we can do more than regularly spaced frequencies.} 

An example of an operator in the combined space is the free space propagation operator:
\begin{align}
\label{eq:Propagator}
\oper{P}(\Delta z,\eta) = & \\
\sum_{\substack{mnkl\\rs\\pq}}& \vect{mn;r;p}
\delta_{mk}\delta_{nl}\delta_{rs}\delta_{pq}
\exp  \left[i(m+n+1)\eta
-i\frac{\omega_0+\omega_r}{c}\Delta z\right] 
\form{kl;s;q}, \notag
\end{align}
using parameters
\begin{align*}
\Delta z &= z_{\text{end}}-z_{\text{start}} \\
\eta &= \arctan \left( \frac{(z_{\text{end}}-z_0)\lambda}{\pi w_0^2}\right)- 
        \arctan \left( \frac{(z_{\text{start}}-z_0)\lambda}{\pi w_0^2}\right)
\end{align*}
for starting and ending positions $z_{\text{start}}$ and $z_{\text{end}}$, where $z_0$ is the position of the beam waist, and $w_0$ is the beam radius at the waist. %NL%
The parameter $\eta$ is commonly refered to as the Gouy phase shift. %NL%
As one can see by the presence of the Kronecker deltas in (\ref{eq:Propagator}), all the components of the input field are directly propagated to the output field without mixing. %NL%
There is, however a phase shift which depends on the tranverse mode order (the Gouy shift) and the wavelength ($\omega_r/c$).

More complicated operators exist which mix simultaneously mode and frequency components. %NL%
One example is that of a mirror which is dithered at a given frequency in angle. %NL%
It is in a sense a combination of the misaligned mirror operator in section \ref{sec:modalspace} and the phase modulation operator in section \ref{sec:freqspace}. %NL%
This operator is derived by \citet{Sigg:00}.
\section{Photodetection}
A photodetector is a device which is sensitive to the power in the laser field. %NL%
This is typically photodiode which converts the laser power into an electric current. %NL%
In terms of the electric field we may write the power as
\begin{equation}
P=\frac{\epsilon_0 c}{2}\int \! %NL%
\! %NL%
\dd t \int \! %NL%
\! %NL%
\dd A|E^2|=\frac{\epsilon_0 c}{2}\inprod{E}{E}.
\end{equation}
When using the vector space formalism, this represents the time averaged power detected by the photodiode. %NL%
The time dependent component of the laser power can be thought of as the mixing of different frequency components of the electric field. %NL%
To achieve this in the vector space formalism, we must use an operator which has matrix elements connecting different frequency components. %NL%
For demodulation at frequency $\omega_d$, the  operator takes the form: 
\begin{equation}
\oper{D}(\omega_d) = \sum_{rs} \vect{r} \delta(\omega_r,\omega_s-\omega_d) \form{s},
\end{equation}
where we use a delta function defined as
\begin{equation}
\delta(x,y) = 
\begin{cases}
0 & \text{if $x \neq y$}, \\
1 & \text{if $x = y$}.
\end{cases}
\end{equation}

\com{power at a given frequency? Make sure it's real valued.}

Photodetectors are sometimes split into many segments, such as is the case with a so called quadrant photo-detector (QPD). %NL%
In the case of a QPD, the face of the photodetector is split into four equal segments as seen in figure \ref{fig:QPD}. %NL%
\com{make QPD figure.} Different linear combinations of the photocurrents of these segments can yield different information about the laser beam. %NL%
For example, the vertical displacement of a beam can be determined by subtracting current of the top segments from the bottom segments.

The shape and linear combination of the photodiode segments is treated in the literature\cite{Hefetz:97} as a pupil function $p(x,y)$, \com{how does hefetz do the units?} where the photodiode measures
\begin{equation}
P=\iint\limits_{\text{PD area}}E^\dagger(x,y)p(x,y)E(x,y)\dd x \dd y,
\end{equation}
so for example for a photodetector split along the $y$ axis, $p(x,y)=1$ for $x>0$ and $-1$ for $x<0$.

% show how this is equivalent to an operator

In the language of tranverse spatial modes, a split photodetector is measuring the mixing of the \TEM{00} mode with the \TEM{10} mode.\footnote{More precisly, it measures the mixing of all even modes to odd modes and vice versa, with the primary contribution coming from the modes which differ by one mode order.} So in order to model this in the vector space framework, we will need to generalize the demodulation operator to have matrix elements between spatial modes as well as frequency components.

% explain the ideal 10 and 20 detectors (what is the ideal 20 detector?) Maybe calculate some matrix elements of a real bullseye detector.
% maybe put the numerical values of split detector into appendix

\section{Application to compicated optical systems}
The real utility of this formalism arises when the optical system contains branches that branch and loop onto themselves, as is the case with resonant cavities. %NL%
The fact that the operators can simultaneously handle the transverse modes and frequency components reduces much of the complexity of the problem through abstraction. %NL%


%a simple ring cavity
