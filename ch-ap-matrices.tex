\chapter{Practical calculations using matrices based on the optical vector space model}

\section{Collapsing a tensor of several indices into a 2D matrix}
The tensors in the full optical vector space model described in Chapter \ref{ch:modalmodel} are tensors with elements that are referenced with with up to eight indicies. %NL%
Dealing with eight-dimensional tensors using typical mathematical software environments, such as \mbox{MATLAB} or Mathematica, may be possible, however a far more simple talk would be to re-index the tensors into two dimensional matrices which are more easily dealt with.

For example, we want to take the elements of the free space propagator $\oper{P}$ in Equation \ref{eq:Propagator}, and reduce the number of indices.
\begin{equation}
\matrixel{mn;r;p}{\oper{P}}{kl;s;q} = P_{mnrpklsq} \Rightarrow P_{ij}.
\end{equation}
Thus we need some function that takes four indices and produces a single collapsed index, and there must be a one-to-one mapping of $klsq$ to $i$ such that a given collapsed index unambiguoisly determines the element in the original vector space. %NL%
The result of such an excersize is to produce a formula similar to Equation (6) in \citet{Sigg:00}, but we will cover some details here, as well as point out an error in their formula!

In the modal space,

% explain equation (6) in dynamical response. I believe there is a missing factor of 2 on a(r)

% you want to translate 4 indicies into 1, and it should be a 1 to 1 mapping

% the modal indices can be traversed diagonally nicely with (m+n)*(m+n+1)/2+n

% to expand, general formula i = p^(r^g[m,n]+r)+p. p^=2

% kron multiplies sub matrices

\section{Define the free space propagator to have no phase rotation for the \TEM{00} carrier.}

% this defines the 'microscopic phase' to be zero, thus 00 carrier mode is always resonant in cavities. The higher order modes get the correct relative Gouy phase and the sidebands get their relative phase rotations. Can define another propagator to have 'microscopic phase' in order to set resonance condition of carrier

\section{Units}

% make the electric field have units of sqrt(W). Essentially choose units where epsilon_0 c / 2 = 1

\section{Calculating signal to noise ratios}
% for pupil function that is all ones, the shot noise doesn't change! that's useful. 

% general pupil functions? try for 5 minutes to calculate this.