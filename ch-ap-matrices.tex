\chapter{Practical calculations using matrices based on the optical vector space model}

\section{Collapsing a tensor of several indices into a 2D matrix}
The tensors in the full optical vector space model described in Chapter \ref{ch:modalmodel} are tensors with elements that are referenced with with up to eight indices. %NL%
Dealing with eight-dimensional tensors using typical mathematical software environments, such as \mbox{MATLAB} or Mathematica, may be possible, however a far more simple talk would be to re-index the tensors into two dimensional matrices which are more easily dealt with.

For example, we want to take the elements of the free space propagator $\oper{P}$ in Equation \ref{eq:Propagator}, and reduce the number of indices.
\begin{equation}
\matrixel{mn;r;p}{\oper{P}}{kl;s;q} = P_{mnrpklsq} \Rightarrow P_{ij}.
\end{equation}
Thus we need some function that takes four indices and produces a single collapsed index, and there must be a one-to-one mapping of $klsq$ to $i$ such that a given collapsed index unambiguously determines the element in the original vector space. %NL%
The result of such an exercise is to produce a formula similar to Equation 6 in \citet{Sigg:00}, but we will cover some details here, as well as point out an error in their formula!

One prescription for traversing the modal space is to cover the $m+n$ orders sequentially. %NL%
Starting with the \TEM{00} mode, then the \TEM{01} and \TEM{10} modes, then \TEM{02}, \TEM{11}, and \TEM{20}, etc. %NL%
One way to do this is to define a collapsed index as
\begin{equation*}
g=(m+n)(m+n+1)/2 + n.
\end{equation*}
This leads to an index that counts diagonally through the modal indices like so:
\begin{center}
 \begin{tabular}{ r c|c c c l}
 \multicolumn{5}{r}{$\overbrace{\rule{1.5cm}{0pt}}^{\mbox{$m$}}$} \\
&  & 0 & 1 & 2 & \\
 \cline{2-5}
\multirow{3}{*}{ $n$ $\left\{ \rule{0pt}{1cm} \right. $}
& 0 & 0 & 1 & 3 & 
\multirow{3}{*}{ $\left. \rule{0pt}{1cm} \right\}$ $g$}\\ 
& 1 & 2 & 4 &   \\ 
& 2 & 5 &   &   \\ 
\end{tabular} 
\end{center}

For the frequency index, $r$, one is usually concerned with symmetric upper and lower spaced sidebands where the frequency separations of paired sidebands from the carrier are given by $\omega_{-r} = -\omega_{r}$ and the index traverses the range $-(n_f-1)/2\le r \le (n_f-1)/2$, where $n_f$ is the number of frequency components. %NL%
The carrier is given by $\omega_0=0$. %NL%
From $r$, we may construct an index with only positive values as follows

\begin{equation*}
a= 
\begin{cases} r, & \text{if $r \ge 0 $,}
\\
r+n_f, &\text{if $r < 0$.}
\end{cases}
\end{equation*}
 This definition of $a$ starts at the carrier at $r=0$, then traverses the positive sidebands, before moving through the negative sidebands in reverse.

One may collapse the modal and frequency indices in the following way
\begin{equation}
\label{eqn:nopol}
i = \hat{a}g + a,
\end{equation}
where $\hat{a}$ is the number of possible values of $a$, which is, in this case, the same as $n_f$. %NL%
This collapsed index behaves in the following way for values of $g$ and $a$
\begin{center}
\begin{tabular}{ r| c c c | c c c}
$g$ & \multicolumn{3}{c|}{0} & \multicolumn{3}{c}{1} \\
\hline
a & 0 & 1 & 2 & 0 & 1 & 2 \\
\hline
i & 0 & 1 & 2 & 3 & 4 & 5
\end{tabular} 
\end{center}
In the preceding case, $n_f=\hat{a}=3$. %NL%
As one can see, each increment of the value of $g$ shifts the $a$ index and allows it to start counting again.

The polarization index, $p$ may be included in the same way, adding another layer of shifting to the other indices. %NL%
There are only 2 polarization components, so $\hat{p}=2$. %NL%
And we will let $p \in \{0,1\}$ for the two polarization states. %NL%
The total collapsed index is then
\begin{equation}
i = \hat{p}(\hat{a}g+a)+p=2n_f\left[\frac{(m+n)(m+n+1)}{2} + n\right]+2a+p.
\end{equation}
This is very close to Equation 6 in Sigg and Mavalvala, though the reference is missing a factor of 2 on the $a$ index. %NL%
If one would rather choose to ignore the polarization index, Equation \ref{eqn:nopol} may be used instead.

As a final note on collapsed indices we will note that the translation of the tensor product operation in the eight dimensional representation is represented by a \emph{Kronecker product} of matrices in the collapsed matrix space. %NL%
For example, ignoring the polarization space, the three operations
\begin{align*}
\oper{G}&=\oper{G}_{\text{modal}}\otimes \oper{G}_{\text{frequency}}\\
G_{mnrkls}&=G_{mnkl}\times G_{rs}\\
G_{ij}&=\operatorname{Kron}(G_{g_1g_2},G_{a_1a_2})
\end{align*}
all represent the sample tensor multiplication operation, in the abstract tensor space, in the six index component representation, and in the collapsed index matrix representation, respectively. %NL%
$\operatorname{Kron}(\cdot,\cdot)$ is the Kronecker product, and $i$, $g_1$, $a_1$ and $j$, $g_2$, $a_2$ are related as in Equation \ref{eqn:nopol}.

\section{Avoid decimal arithmetic bugs}
The frequency mixing delta function in the demodulation operator, Equation \ref{eqn:fdemodoperator}, takes the difference of two frequencies and compares that to another frequency. %NL%
In a numerical calculation, this can lead to bugs when the frequency values cannot be exactly represented as a binary number. %NL%
To avoid this, one may allow the delta function to take the value 1 when the frequencies being compared are closer than some threshold.

\section{Define the free space propagator to have no phase rotation for the \TEM{00} carrier.}

It is usually desired that when modeling resonant cavities, the default behavior in the steady state is for the carrier frequency component of the laser to be resonant in the cavity. %NL%
In the real world of cavities and lasers, this resonance condition requires a very careful tuning of the length of the cavity, usually in the form of a control system. %NL%
In the modeled world, however, we may take a convenient shortcut to ensure that the carrier light is resonant by default in any cavity we construct.

The free space propagator operator (Equation \ref{eq:Propagator}) contains a phase rotation common to all components of the field. %NL%
If one were to remove this phase rotation by construction, by letting $\exp[i(\eta - \omega_0 \Delta z/c)]=1$, this would maintain the correct relative phase rotation of all the other components relative to the carrier, without rotating the carrier. %NL%
In a sense this defines the `microscopic phase' to be zero, and thus the \TEM{00} carrier component is always resonant in cavities. %NL%
One may still model detunings of the carrier from resonance by defining a microscopic propagator which applies the correct detuning phase to the carrier.

\section{Units}
It is useful to choose a set of units where $\epsilon_0 c/2=1$. %NL%
This essentially chooses the field components to have units of $\rm{W}^{\frac{1}{2}}$. %NL%
This simplifies the use of Equation \ref{eqn:sigdemodoper}.

\section{Calculating signal to noise ratios}
In the language of the vector space model, the shot noise of a full plane photodetector is 
\begin{equation}
\label{eqn:modalshotnoise}
N=\sqrt{2\hbar \omega_0 \inprod{E}{E}},
\end{equation}
where $\omega_0$ is the absolute optical angular frequency of the carrier.

In the case of a split detector the situation is the same. %NL%
As long as the pupil function is $-1$ or $1$ everywhere on the beam, the noise contributes incoherently and with equal weight everywhere, giving again Equation \ref{eqn:modalshotnoise}. %NL%
The situation of the noise for any generic pupil function is left as an exercise for the motivated reader.
