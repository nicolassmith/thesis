\chapter{Gravitational waves}
\label{ch:gws}
A story bordering on apocryphal states that Albert Einstein made two predictions about gravitational waves in 1916. %NL%
First, that they exist, and second, that they were certainly too small for their effects to ever be directly measured. %NL%
It is the hope of the author that in the coming years Einstein will be proved wrong for the second of these two predictions.

This chapter covers a very brief overview of the theory of gravitational waves. %NL%


% what is the metric
\section{Gravity according to Einstein}
In 1915, Einstein described space-time as a four dimensional manifold which may be completely described by a metric tensor $\oper{g}$. %NL%
The space-time metric defines a co-variant interval between infinitesimally separated events as
\begin{equation}
\dd s^2 = g_{\mu \nu}\dd x^\mu \dd x^\nu,
\end{equation}
where $\mu$ and $\nu$ are indicies for the space and time coordinates, and $g_{\mu \nu}$ are the elements of $\oper{g}$. %NL%
The trajectories of free particles are defined by geodesic paths in the manifold, which may be derived from the metric tensor\cite[Chap. %NL%
3]{carroll2004spacetime}.

The metric satisfies the sourced field equation
\begin{equation}
\label{eqn:einstein}
\oper{G}=\frac{8\pi G}{c^4}\oper{T},
\end{equation}
where $\oper{G}$ is known as the Einstein tensor, a second-order differential operator acting on the metric tensor, $G$ is Newton's constant, $c$ is the speed of light, and $\oper{T}$ is the stress-energy tensor which holds information about all forms of energy in the space-time.

\section{The general theory of relativity in the linearized regime}
In the weak field limit of General Relativity, the metric tensor $\oper{g}$ will be close to the Minkowski metric of Special Relativity \cite[Chap. %NL%
7]{carroll2004spacetime}. %NL%
Far from sources of gravitational waves, the wave amplitudes will be quite small and thus
\begin{align}
g_{\mu \nu}&\approx \eta_{\mu \nu}+h_{\mu \nu}, &|h_{\mu \nu}|\ll 1.
\end{align}
where $\eta_{\mu \nu}$ are the components of the Minkowski metric, and $h_{\mu \nu}$ are small pertubations.

We are interested in Wave-like solutions that will propagate in the absence of any source terms (although they may be created by sources), thus $\oper{T}=0$. %NL%
For convenience we choose what is known as the \emph{transverse traceless} gauge. %NL%
Given these simplifications, Equation \ref{eqn:einstein} becomes
\begin{equation}
\left(-\frac{1}{c^2}\frac{\partial^2}{\partial t^2}+\nabla^2\right)h_{\mu \nu}=0.
\end{equation}
This is of course the wave equation in three dimensions, which has solutions of the form
\begin{equation}
\oper{h}=\oper{A}\exp{(ik_\alpha x^\alpha)}.
\end{equation}
Due to symmetries and chosen gauge conditions, a wave traveling in the $z$ direction with amplitude $\oper{A}$ has only two independent components \cite[Chap. %NL%
9]{schutz1985first}
\begin{align*}
A_{xx}&=-A_{yy}\equiv h_+,\\
A_{xy}&=A{yx}\equiv h_\times,
\end{align*}
the rest being zero. %NL%
Thus such a wave takes the form
\begin{equation}
\label{eqn:wavefunc}
\oper{h} \doteq 
\ms
0 & 0 & 0 & 0 \\
0 & h_+ & h_\times & 0 \\
0 & h_\times & -h_+ & 0 \\
0 & 0 & 0 & 0
\me
\cos(\omega [z/c- t]).
\end{equation}
Here we explore one of the physical observables of such a wave.

\section{The phase of a pulse of light in the presence of a gravitational wave}
Presume we have a device that can measure the phase of an electromagnetic wave. %NL%
The device is not under the influence of any external forces apart from gravity. %NL%
Suppose we send a pulse of light in a particular direction, we have arranged for the pulse to return to us with a system of mirrors and transceivers. %NL%
The total phase measured (modulo $2\pi$) of a light pulse with frequency $f$ after traveling along a path $\rm S$ is\footnote{The treatment given in this chapter assumes that the path taken by the light pulse is short compared to the wavelength of the gravitational wave, known as the `long wavelength approximation.' A fully consistent relativistic treatment is given by \citet{RakhmanovPhoton}.}
\begin{equation}
\phi_{\text{round trip}} = 2\pi f\oint_{\rm S} \dd t.
\end{equation}
Along the path, the pulse follows a null trajectory ($\dt s^2=0$) and furthermore, according to Equation \ref{eqn:wavefunc}, the metric does not mix time and spatial components ($g_{0i}=g_{i0}=0$), thus
\begin{equation}
c\dd t = \sqrt{|g_{ij}|\dd x^i \dd x^j},
\end{equation}
where $i$ and $j$ only include spatial components.

Let us consider a simple path made of two segments, propagation from the origin $(0,0,0)$ to a point in the $x$ direction which is $L_x$ away $(L_x,0,0)$, and then back to the origin. %NL%
The phase along this path is
\begin{equation}
\phi_{\text{round trip}}^{\text{x-path}}= \frac{2\pi f}{c} \left(\int_0^{L_x}-\int^0_{L_x}\right)\sqrt{|g_{xx}|}\dd x \approx \frac{4\pi f}{c} L_x \left(1+\frac{h_+}{2}\right).
\end{equation} 
The same argument for a path along the $y$ gives
\begin{equation}
\phi_{\text{round trip}}^{\text{y-path}} \approx \frac{4\pi f}{c} L_y \left(1-\frac{h_+}{2}\right).
\end{equation} 

Instead, we may choose a path along the line $y=x$, in this case integration is taken along $ u(\sigma) = (\sigma,\sigma,0)$. %NL%
The phase obtained is
\begin{equation}
\phi_{\text{round trip}}^{\text{u-path}} \approx \frac{4\pi f}{c} L_u \left(1+\frac{h_\times}{2}\right).
\end{equation} 
Thus the effect of the $h_+$ amplitude is to differentially stretch paths in the $x$ and $y$ directions, while $h_\times$ stretches along axes $45 \degrees$ from $x$ and $y$. %NL%
 The stretching effect is proportional to the unstretched length, and thus is manifested as a strain.
\com{figure of ring of masses}
\section{Sources of gravitational waves}
Of the known physical forces, gravity is by far the most feeble. %NL%
When Einstein first postulated the existence of gravitational waves, he thought their effects would never be detected.

According to General Relativity, the lowest mass-energy multipole moment that may source gravitational radiation is the mass quadrupole. %NL%
The lower moments, the mass-energy monopole and dipole are guaranteed constant by energy and momentum conservation, respectively \cite[Section 7.5]{carroll2004spacetime}.

A system of two bodies in orbit is a splendid source of gravitational radiation. %NL%
The first evidence of the existence of gravitational radiation came from the Hulse-Taylor pulsar system which showed the characteristic loss of orbital energy due to gravitational wave radiation damping \cite{Taylor1979}. %NL%
The waves emitted by this system will have a discouragingly small amplitude of $h\approx 10^{-26}$ when they reach Earth, not to mention a period of several hours. %NL%
This system will decay in a time of order $10^8$ years.

A compact binary system such as the Hulse-Taylor binary in the final orbits before merger provides a more promising prospect for the creation of gravitational waves that may be measurable on Earth. %NL%
The time scales of such collisions would lead to gravitational waves in the audio frequency band. %NL%
Of the expected sources, the coalescence of black holes or neutron stars are the ones most likely to produce gravitational wave with enough magnitude to be detected on Earth with modern detectors \cite{CBCrates}. %NL%
Data taken from the LIGO and Virgo detectors have been analyzed and upper limits of these sources have been set \cite{CBC,CBCblackholes}.

It is also possible that an aspherical spinning neutron star will produce waves large enough to be detected. %NL%
Such waves would be largely monochromatic, with some modulation due to relative motion of the source and the Earth. %NL%
Currently, the best limit on the contribution of neutron star spin-down of the Crab\cite{Crab} and Vela\cite{Vela} pulsars are determined by the lack of gravitational radiation detected by LIGO and Virgo.

There are some transient gravitational wave sources for which no waveform models exist. %NL%
An example of such a source would be the waves emitted by a supernova explosion. %NL%
Searches for such unmodeled sources have yielded a strain sensitivity of $\sim 10^{-21} / \sqrt{\rm{Hz}}$, but no waves have yet been detected \cite{bursts}.

We may one day discover a stationary, broadband source of gravitational radiation. %NL%
Most cosmological model predict some level of stochastic gravitational wave background \cite{bbn,stochdirectional}. %NL%
The information gained by detecting such waves will be extremely interesting because if they are of cosmological origin, they would have been created mere fractions of a second after the Big Bang.

Now we focus on the antennae used to detect gravitational radiation.
