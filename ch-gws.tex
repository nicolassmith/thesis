\chapter{Gravitational waves}

% einstein describes GWs
\com{Einstein and GR.}

% what is the metric
Einstein described space-time as a four dimensional manifold which may be completely described by a metric tensor $\oper{g}$. %NL%
The space-time metric defines a co-variant interval between infinitely seperated events as
\begin{equation}
\dd s^2 = g_{\mu \nu}\dd x^\mu \dd x^\nu,
\end{equation}
where $\mu$ and $\nu$ are indicies for the space and time coordinates, and $g_{\mu \nu}$ are the elements of $\oper{g}$. %NL%
The trajectories of free particles are defined by geodesic paths in the manifold, which may be derived from the metric tensor.

The metric satisfies the sourced field equation
\begin{equation}
\label{eqn:einstein}
\oper{G}=\frac{8\pi G}{c^4}\oper{T},
\end{equation}
where $\oper{G}$ is known as the Einstein tensor and acts as a second-order differential operator on the metric tensor, $G$ is Newton's constant, $c$ is the speed of light, and $\oper{T}$ is the stress-energy tensor which holds information about all forms of energy in the space-time.\footnote{Further reading on the General Theory of Relativity can be found in \com{Schultz and Carrol}}

\section{The general theory of relativity in the linearized regime}

Wave-like solutions to Equation \ref{eqn:einstein} \com{finish}. %NL%
We are interested in finding \com{finish}.

\com{gravitational waves}
\com{get wave equation and the h+ and hx polarizations}

\section{The phase of a photon in the presence of a gravitational wave}
Presume we have a device which can measure the phase of a plane electromagnetic wave or alternatively, a photon. %NL%
The device is not under the influence of any external forces apart from gravity. %NL%
Suppose we send a photon in a particular direction, we have arranged for the photon to return to us with a system of mirrors and transceivers. %NL%
The total phase measured (modulo $2\pi$) of a photon with frequency $f$ after traveling along a path $\rm S$ is\footnote{The treatment given in this chapter assumes that the path taken by the photon is short compared to the wavelength of the gravitational wave, known as the `long wavelength approximation.' A fully consisten relativistic treatment is given by \citet{RakhmanovPhoton}.}
\begin{equation}
\phi_{\text{round trip}} = 2\pi f\oint_{\rm S} \dd t.
\end{equation}
Along the path, the photon follows a null trajectory ($\dt s^2=0$) and furthermore in \com{equation above} the metric does not mix time and spatial components ($g_{0i}=g_{i0}=0$), thus
\begin{equation}
c\dd t = \sqrt{g_{ij}\dd x^i \dd x^j},
\end{equation}
where $i$ and $j$ only include spatial components.

Let us consider a simple path made of two segments, propagation from the origin $(0,0,0)$ to a point in the $x$ direction which is $L_x$ away $(L_x,0,0)$, and then back to the origin. %NL%
The phase along this path is
\begin{equation}
\phi_{\text{round trip}}^{\text{x-path}}= \frac{2\pi f}{c} \left(\int_0^{L_x}-\int^0_{L_x}\right)\sqrt{g_{xx}}\dd x \approx \frac{4\pi f}{c} L_x \left(1+\frac{h_+}{2}\right).
\end{equation} 
The same argument for a path along the $y$ gives
\begin{equation}
\phi_{\text{round trip}}^{\text{y-path}} \approx \frac{4\pi f}{c} L_y \left(1-\frac{h_+}{2}\right).
\end{equation} 

Instead, we may choose a path along the line $x=y$, in this case integration is taken along $ u(\sigma) = (\sigma,\sigma,0)$. %NL%
The phase obtained is
\begin{equation}
\phi_{\text{round trip}}^{\text{u-path}} \approx \frac{4\pi f}{c} L_u \left(1+\frac{h_\times}{2}\right).
\end{equation} 
Thus the effect of the $h_+$ amplitude is to differentially stretch paths in the $x$ and $y$ directions, while $h_\times$ stretches along axes $45 \degrees$ from $x$ and $y$. %NL%

\com{figure of ring of masses}
\section{Sources of gravitational waves}
% just cite some source papers, also maybe give the canonical 10^-21 thing. No calculations.

