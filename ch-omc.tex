\chapter{The output mode cleaner}

\section{Motivation for an OMC}

\section{Optical and mechanical design of the OMC}

\section{Characterization of the H1 OMC}

\subsection{Cavity FSR and higher order mode spacing}

\com{Show diagram of OMC interrogator, and desribe experimental setup.}

Figure \ref{fig:OMCintdiagram} shows a diagram of the experimental setup used to measure the Free Spectral Range (FSR) and higher order mode spacing of the H1 OMC. %NL%
The laser source is a NdYAG NPRO providing a 1064nm wavelength laser beam. %NL%
A beam splitter sends some fraction \com{how much} of the light through an electro-optic modulator (EOM) which is driven at by RF oscillator \#1. %NL%
This introduces two RF sidebands which will be used for cavity length control. %NL%
The other path of light is director to an acouso-optic modulator (AOM). %NL%
The light is double passed though the AOM and and receives a frequency shift which is twice the frequency of RF oscillator \#2. %NL%
The light from the two paths are recombined before being injected into an optical fiber. %NL%
The frequency makeup of the combined beam includes the original carrier frequency, two RF sidebands and a frequency shifted subcarrier.

The light exiting the other end of the optical fiber is incident on the input coupling mirror of the OMC cavity. %NL%
The promptly reflected beam is detected on a photodetector where the photocurrent is demodulated at the frequency of RF oscillator \#1. %NL%
Choosing the correct demodulation phase provides a laser frequency error signal which is fed back to the frequency actuator of the laser. %NL%
The control system is able to maintain resonance of the laser in the OMC. %NL%
The laser transmitted through the OMC is detected on a second photodetector, as well as on a CCD sensor.

Measurement of the cavity FSR is acheived by varying the frequency of the subcarrier and measuring the OMC transmitted power. %NL%
The central maximum of the cavity resonance will occur when the shifted subcarrier frequency is equal to the FSR.

Similarly, the higher order mode frequency shift is measured by varying the subcarrier frequency until the subcarrier is resonant on a higher order mode of the OMC. %NL%
Coupling of the subcarrier beam into the HOMs is enhanced if slight misalignments are introduced on the input beam. %NL%
The order of the mode is determined by the image recorded by the CCD in transmission.

\com{data here.}

These data were taken with the OMC at room temperature. %NL%
It was discovered that the HOM spacing varied with the temperature of the thermal length actuator. %NL%
\com{Talk more about this? %NL%
it's own short section?}

\subsection{Finesse}

The frequency profile of the transmitted peak when the subcarrier is shifted through resonance may be used to determine the cavity Finesse. %NL%
The transmission as a function of frequency is
\begin{equation}
1+1=2.
\end{equation}

\com{data here.}

\subsection{Cavity losses}
The intra-cavity loss of the OMC was infered by using 3 photodetectors. %NL%
One measuring a sample of the input light, one measuring the reflected light, and one measuring the transmitted light. %NL%
Care was taken to ensure that the three photodetectors were calibrated relative to each other.

\com{data here.}

\subsection{PZT actuator response}
The PZT length actuator of the OMC was calibrated using by using the PZT to sweep the cavity through both carrier and subcarrier resonances while recording the PZT voltage. %NL%
The frequency spacing of the carrier and subcarrier is used to calibrated the PZT length actuation coefficient.

\com{PZT sweep figure here.}

The frequency response of the PZT actuator was measured at the error signal of the cavity frequency locking loop while driving the OMC PZT. %NL%
The effects of the control loop have been removed from the data.

\com{TF here.}

\section{Opto-thermal-mechanical feedback issues in the OMC}

\com{At least include measurements by Tim and I, hopefully can hack a model to explain it.}

%%% Local Variables: 
%%% mode: latex
%%% TeX-master: "main"
%%% End: 
